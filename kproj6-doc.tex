\documentclass[11pt,a4paper,titlepage,oneside]{report}

\usepackage[english]{babel}
\usepackage[utf8]{inputenc} % input encoding is UTF-8

\usepackage{graphicx}
\usepackage{tabularx}

\usepackage{color}
\usepackage[unicode,pdftex]{hyperref}

\begin{document}

% Title page %%%%%%%%%%%%%%%%%%%%%%%%%%%%%%%%%%%%%%%%%%%%%%%%%%%%%%%%
\begin{titlepage}

\begin{center}
\includegraphics[width=0.5\textwidth]{img/logo_NTNU.png}\\
\vfill
{\LARGE \textbf{TDT4290 - Customer Driven Project}}
\vfill
{\Huge \textbf{Ocean forecast}}

\vspace{12pt}
{\LARGE \textbf{SINTEF}}

\vspace{30pt}
{\LARGE \textbf{Pre study}}
\vfill
{\LARGE \textbf{Autumn 2014}}
\end{center}
\vfill
\begin{tabular*}{\textwidth}{@{\extracolsep{\fill}} l l}
\textbf{Group 6} & \textbf{Advisor} \\
Arve Nygård & Gleb Sizov \\
Anders Smedegaard Pedersen & \\
Emil Jakobus Schroeder & \\
Hans Kristian Henriksen & \\
Marco Radavelli & \\
Ondřej Hujňák & \\
Ruben Håskjold Fagerli & \\
\end{tabular*}

\end{titlepage}

% Empty page %%%%%%%%%%%%%%%%%%%%%%%%%%%%%%%%%%%%%%%%%%%%%%%%%%%%%%%%
\newpage
\thispagestyle{empty}
\mbox{}
\newpage

% Abstract %%%%%%%%%%%%%%%%%%%%%%%%%%%%%%%%%%%%%%%%%%%%%%%%%%%%%%%%%%
\begin{abstract}
Na na na na na na na, Batman!!!
\end{abstract}

% Signatures %%%%%%%%%%%%%%%%%%%%%%%%%%%%%%%%%%%%%%%%%%%%%%%%%%%%%%%%
\thispagestyle{empty}
\begin{center}
{\large \textbf{Trondheim, \today}}\\
\vspace{2.5cm}
\begin{tabularx}{\textwidth}{@{\extracolsep{1cm}} X X }
\dotfill & \dotfill \\
~Arve Nygård & ~Anders Smedegaard \\[1cm]
\dotfill & \dotfill \\
~Emil Jakobus Schroeder & ~Hans Kristian Henriksen \\[1cm]
\dotfill & \dotfill \\
~Marco Radavelli & ~Ondřej Hujňák \\[1cm]
\dotfill & \\
~Ruben Håskjold Fagerli & \\[1cm]
\end{tabularx}
\end{center}

% Table of contents %%%%%%%%%%%%%%%%%%%%%%%%%%%%%%%%%%%%%%%%%%%%%%%%%
\tableofcontents
\addtocontents{toc}{\protect\thispagestyle{empty}}

% List of figures %%%%%%%%%%%%%%%%%%%%%%%%%%%%%%%%%%%%%%%%%%%%%%%%%%%
\listoffigures
\addtocontents{lof}{\protect\thispagestyle{empty}}

% List of tables %%%%%%%%%%%%%%%%%%%%%%%%%%%%%%%%%%%%%%%%%%%%%%%%%%%%
\listoftables
\addtocontents{lot}{\protect\thispagestyle{empty}}

\pagenumbering{arabic}
\setcounter{page}{0}

% Main body %%%%%%%%%%%%%%%%%%%%%%%%%%%%%%%%%%%%%%%%%%%%%%%%%%%%%%%%%
%%%%%%%%%%%%%%%%%%%%%%%%%%%%%%%%%%%%%%%%%%%%%%%%%%%%%%%%%%%%%%%%%%%%%

\chapter{Introduction}
\section{TDT 4290 - Customer driven project}
The task is set forth in the subject TDT 4290 - Customer driven project at the Norwegian university of science and technology. The goal of the course is 
\begin{quote}
(...)to give the students a practical experience of carrying out all the phases of a typical customer guided IS/IT-project.\cite{TDT4290 intro}
\end{quote}
The subject divides the students into random groups, and assigns each group an assignment. The assignments are real problems that businesses needs solved. 

Although the assignment is to follow the entire process of an IT-project, the focus is on the earlier phases of a project. Thus, an important part of the assignment is the work leading up to the implementation phase.

\section{Pre study}
To get an understanding of the customers needs, as well as studying different solutions, a pre study is conducted. From the course compendium:
\begin{quote}
The preliminary studies are vital for the group to obtain a good understanding of the total problem.
Here, you will have to describe the problem at hand. You should describe the current system and the
planned solutions (...).
\end{quote}

The report is produced to formalise the recommendation that is made to the customer. Although the report contains the entire background and recommendation, the customer has been kept up to date with the work, and been presented with our conclusions well before this study was finished. This has been necessary for our work to continue, and was agreed upon with the customer in advance.

\section{Work organisation}
In this first work period, we have worked in two phases. For the first phase, we worked with the goal of familiarising ourselves with the file format, and the technology. In this period the group was divided into two groups, one investigating the front end, and one the back end. For the second phase, the group adopted the scrum methodology, and used one sprint to finish the recommendation. 


\chapter{Assignment and use cases}
This chapter will give a brief description of the assignment given to the group, the problem domain, as well as some example use cases. 
\section{Domain}
The groups assignment is connected to SINTEFs work on oceanographic simulations. By analysing large amounts of observational data, and running this trough complex simulations, the goal is to predict future conditions. 

More specifically, the currents, temperature, salinity and other factors are recorded, simulated and predicted. These predictions are used mainly by stakeholders in the fish farming industry for decision support. Usually this is in conjunction with lice removal, which requires precise positioning of large supply vessels. 

\section{Assignment}
The assignment set forth by SINTEF is to improve the current solution to be able to serve the user with dynamically created views of the data. 
\section{Use cases}

\chapter{Current situation}
In this chapter we will explore the solution SINTEF is currently using, and the challenges and limitations it poses. After looking into this, we will describe the evaluation criteria that will be used to assess the alternative solutions the group has found. 
\section{Current system}
<Figure of SinMod landing page for location>
\\
\\
The current system deployed at SINTEF serves their clients by providing access to a collection of more than 100 000 pre generated PDF files. These files contain information on currents, salinity, and temperature. The user may choose what information he wants by selecting parameters in the drop down menus. 

If the user chooses a specific site from the map or location drop down, he will be presented with key data for this area. This includes statistical information such as maximum current speed, average current speed and so forth, as well as geographical position. 

The system will also present a set of pre defined graphs, including current roses, tidal ellipsis and vertical profiles. The graphs are given for standard attributes (e.g depth 2 meters), and for some of them, there is an option of downloading a PDF containing graphs for other values of the given attributes. 

\section{Challenges}
As the PDFs are pre generated, there is a clear limitation to what information the user may request. If a user wants to know, for example, the connection between salinity and current speed at a given location, the user must download two different PDFs and manually compare these. 

The graphs given for a specific location are only given for limited values of the critical attributes. If we look at the current rose, it is presented for a depth of 2 meters. If the user is really interested in the current rose for 10 meters, he has to download the PDF containing all possible current roses. 

The same is true for the maps that can be generated for a specific site. The user may choose period (a single month may be selected), and one of the five variables. This gives the user a PDF with one map for each depth that can be calculated. 

For a user that knows what data is interesting, this is a complicated and data heavy way of delivering information. The PDFs seems to range in size from 125kB to around 3MB, depending on what information is requested. The region maps are the absolute largest in file size, ranging from 1 to 3MB, while the files containing the current roses are quite small, in the 100-200kB range. 

On a computer with broadband connection, the size of the files is not very problematic. For these users, the biggest challenge is the fact that the user can not specify what kind of data they want plotted, and have to look through quite a lot of pages to get the information needed. For a user on a low bandwidth connection and/or on a mobile device, the size of the files is a more pressing problem. On an EDGE connection the theoretical best download time for a 3MB file is 62,5 seconds at 384kbit/s.
http://www.3gpp.org/technologies/keywords-acronyms/102-gprs-edge

\section{Evaluation criteria}

\chapter{Possible solutions}
The group has used the first part of the project investigating what solutions would best suit SINTEFs needs. In the chapter we present the different solutions we have found, along with an assessment on how each solution is rated in accordance to the evaluation criteria. 
\section{Commercial solutions}
\section{Open source solutions}
\section{Custom solutions}

\chapter{Recommendation}
Based on the study of possible solutions in the previous chapter, the group will give a recommendation.


% Sources cited in the document
% uncomment when there are some citations, uncomment bibtex in Makefile
\chapter{Bibliography}
\bibliographystyle{plain}
\begin{flushleft}
	\bibliography{source_library}
	\bibitem[1]{TDT4290 intro} BLALBALBLABLBA
\end{flushleft}

% Appendixes
\appendix

\end{document}
